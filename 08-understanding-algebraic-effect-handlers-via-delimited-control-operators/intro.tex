\section{Introduction}
\label{sec:intro}

Algebraic effects~\cite{plotkin-effect} and handlers~\cite{plotkin-handler}
have become an essential element of a programmer's toolbox.
Effect handlers provide a convenient interface for defining and composing
effects.
They also enable concise implementation of sophisticated behavior by giving
the programmer access to continuations.

Over the past decade, researchers have been actively studying the theory
of effect handlers.
As an outcome of these studies, we have obtained various program
transformations for effect handlers, which can be used to compile effect
handlers into plain
$\lambda$-terms~\cite{hillerstrom-cps,schuster-capability,xie-evidently}.
There are also a variety of type systems for effect handlers, in which
effects are represented as sets~\cite{bauer-effsys},
rows~\cite{hillerstrom-links}, or
capabilities~\cite{brachthauser-lightweight}.

We continue the study of effect handlers, but from a different point of view.
Instead of directly developing the theory of effect handlers, we
\textit{derive} it from the theory of delimited control
operators~\cite{danvy-abstracting,kameyama-dynamic,materzok-subtyping,dyvbig-monadic}.
Control operators have a longer history than effect handlers, and their
theory is closely connected to that of effect handlers~\cite{forster-jfp,pirog-typedeq}.
We aim to improve the understanding of effect handlers by exploiting the 
existing results about control operators, as well as the connection between 
effect handlers and control operators.

In this paper, we discuss a variant of control operators known as \shiftztt
and dollar~\cite{materzok-hierarchy}, and a variant of effect handlers that
are called \textit{deep} handlers~\cite{kammar-handler}.
Our goal is to answer the following research questions.

\begin{itemize}
\item The dollar operator extends the more traditional \resetztt operator
  with a return clause, and this extension is known to cause no change in
  the expressiveness~\cite{materzok-hierarchy}.
  Does this result also apply to effect handlers with and without a return 
  clause?

\item The \shiftztt and dollar operators are associated with a CPS
  translation that can be viewed as a definitional interpreter.
  What CPS translation can we derive for deep effect handlers from the CPS
  translation for \shiftztt and dollar?

\item The typing of \shiftztt and dollar is directed by their CPS
  translation~\cite{danvy-context,kameyama-dynamic,materzok-subtyping},
  rather than their direct-style form.
  What type system can we derive for deep effect handlers using the CPS
  approach?
\end{itemize}

Note that we do not intend to produce results that are particularly 
surprising, nor do we aim to develop something that has immediate 
practical use.
Instead, we would like to explore the possibility of transferring 
knowledge between different control facilities.
This method can be applied for various purposes, and we feel that it is 
especially effective when used in a pedagogical context.

In the rest of this paper, we first define a calculus of \shiftztt/dollar
(Section~\ref{sec:lambdasz}) and another calculus of deep effect handlers
(Section~\ref{sec:lambdah}).
We next answer the three questions one by one (Sections~\ref{sec:return} to
\ref{sec:type2}).
We then discuss related work (Section~\ref{sec:related}) and conclude
with future perspectives (Section~\ref{sec:conclusion}).
